%! Author = João Mendes
%! Date = 08/06/21

% Preamble
\documentclass[a4paper,12pt]{article}

\usepackage{blindtext}
\usepackage{enumitem}
\usepackage{index}
\usepackage{graphicx}
\usepackage[portuguese]{babel}
\usepackage{indentfirst}


\makeindex

\title{Relatório Football Manager}
\author{grupo 23: \\ Francisco Paiva, a93311 \\ João Mendes, a93256 \\ Ricardo Silva, a93195}
\date{\today}
\pagenumbering{arabic}

\begin{document}
    \maketitle

    \begin{center}
    \centering
     %\subfloat [paiva]% para centralizarmos a figura
     \vspace*{1cm}
     \includegraphics[width=2cm]{Photo.aspx2.jpeg}
     \caption{a93311}
     %\label{figura}
     }
     \quad
     \centering
    % \subfloat[mendes]
     \vspace*{1cm}
     \includegraphics[width=2cm]{Photo.aspx1}
     \caption{a93256}
    % \label{figura}
     }
     \quad
     \centering
     %\subfloat[rickz]
     \vspace*{1cm}
     \includegraphics[width=2cm]{Photo.aspx}
     \caption{a93195}
     %\label{figura}
     \end{center}

\begin{center}
    \vspace*{2cm}
    \includegraphics[width=5cm]{EEUMLOGO.png}
\end{center}

\newpage

\tableofcontents

\setcounter{page}{2}

\newpage

\section{Introdução}

\section{Jogador}
    Na Classe Jogador desenvolvemos as seguintes variáveis:
    \begin{itemize}
        \item nome - que se refere ao nome do jogador
        \item nCamisola - que se refere ao numero da camisola
        \item posicao - que indica a posiçao a que joga o jogador. Para sabermos a posiçao do jogador desenvolvemos
um \emph{enum}
        \item skills - que se refere as habilidades do jogador, elaborado através \emph{Map}, onde a \emph{key}
que se refere às Habilidades e a \emph{value} se refere à qualidade da habilidade
        \item historial - que se refere ao historial do jogador, através de uma \emph{List} de equipas por onde o
jogador jogou
    \end{itemize}

    Elaboramos os seus respetivos construtores bem como os seus \emph{getters} e \emph{setters} devido ao encapsulamento.
Nesta classe, desenvolvemos um método que permite calcular o overall do jogador consoante a sua posição. Construimos
ainda um método \emph{ Jogador parse} ...\par

    Por fim, estão definidos as funções \emph{toString}, \emph{equals} e \emph{clone}. \par

\section{Equipa}
    Na Classe Equipa desenvolvemos as seguintes variáveis:
    \begin{itemize}
        \item nome - que se refere ao nome da equipa
        \item fundacaoEquipa - utilizamos um \emph{LocalDate} para se referir a data de fundaçao da equipa
        \item jogadores - que se refere a lista de jogadores que cada equipa tem
    \end{itemize}



\section{Jogo}
    Na Classe Jogo desenvolvemos as seguintes variáveis:
    \begin{itemize}
        \item tempo - que se refere o tempo do jogo
        \item estado - que se refere ao estado do jogo
        \item equipa1 - que se refere à equipa visitada
        \item equipa2 - que se refere à equipa visitante
        \item golosVisitada - que se refere aos golos da equipa visitada
        \item golosVisitante - que se refere aos golos da equipa visitante
        \item jogadoresEquipa1 - que se refere aos titulares da equipa1
        \item jogadoresEquipa2 - que se refere aos titulares da equipa2
        \item taticaEquipa1 - que se refere a tática da equipa visitada
        \item taticaEquipa2 - que se refere a tática da equipa visitante
        \item substituicoesEquipa1 - que se refere às substituições da equipa visitada
        \item substituicoesEquipa2 - que se refere às substituições da equipa visitante
    \end{itemize}

\section{Informacoes}
    Na Classe Informacoes desenvolvemos as seguintes variáveis:
    \begin{itemize}
        \item equipa - que se refere a todas as equipas
        \item jogadores - que se refere a todos os jogadores
        \item jogos - que se refere a todos os jogos
    \end{itemize}


\section{Parser}

\section{Controller}

\section{View}
    Esta classe é encarregue de imprimir toda a informação para o utilizador

\section{FM}
    Este objeto tem como objetivo compilar e correr o programa
    %meter o diagrama para apresentarmos as ligaçoes
\section{Conclusão}


\end{document}
