%! Author = João Mendes
%! Date = 08/06/21

% Preamble
\documentclass[a4paper,12pt]{article}

\usepackage{blindtext}
\usepackage{enumitem}
\usepackage{index}
\usepackage{graphicx}
\usepackage[portuguese]{babel}
\usepackage{indentfirst}
\usepackage{multicol}


\makeindex

\title{Relatório Football Manager}
\author{grupo 23: \\ Francisco Paiva, a93311 \\ João Mendes, a93256 \\ Ricardo Silva, a93195}
\date{\today}
\pagenumbering{arabic}

\begin{document}
    \maketitle

    \begin{multicols}{3}
        \includegraphics[width=3cm]{Photo.aspx2.jpeg}
        \textbf{Fig1: a93311}

        \includegraphics[width=3cm]{Photo.aspx1}
        \textbf{Fig2: a93256}

        \includegraphics[width=3cm]{Photo.aspx}
        \textbf{Fig3: a93195}

    \end{multicols}

\begin{center}
    \vspace*{2cm}
    \includegraphics[width=6cm]{EEUMLOGO.png}
\end{center}

\newpage

\tableofcontents

\setcounter{page}{2}

\newpage

\section{Introdução}
    No âmbito da unidade curricular Programação Orientada aos Objetos, o presente projeto consiste na realização do
\emph{Football Manager}. Desta forma, elaboramos varios objetos tais como: Jogador, Equipa, Jogo, Informacoes, Parser,
Controller, View e Fm, com o intuito de respeitar a modularidade e torna-lo de fácil compreensão.

\section{Jogador}
    Na Classe Jogador desenvolvemos as seguintes variáveis:
    \begin{itemize}
        \item nome - que se refere ao nome do jogador
        \item nCamisola - que se refere ao numero da camisola
        \item posicao - que indica a posiçao a que joga o jogador. Para sabermos a posiçao do jogador desenvolvemos
um \emph{enum}
        \item skills - que se refere as habilidades do jogador, elaborado através \emph{Map}, onde a \emph{key}
que se refere às Habilidades e a \emph{value} se refere à qualidade da habilidade
        \item historial - que se refere ao historial do jogador, através de uma \emph{List} de equipas por onde o
jogador jogou
    \end{itemize}

    Elaboramos os seus respetivos construtores bem como os seus \emph{getters} e \emph{setters} devido ao encapsulamento.
Nesta classe, desenvolvemos um método que permite calcular o overall do jogador consoante a sua posição. Construimos
ainda um método \emph{ Jogador parse} que recebe como argumento uma linha jogador do ficheiro logs e faz a conversão para
o objeto.\par
    Por fim, estão definidos as funções \emph{toString}, \emph{equals} e \emph{clone}. \par

\section{Equipa}
    Na Classe Equipa desenvolvemos as seguintes variáveis:
    \begin{itemize}
        \item nome - que se refere ao nome da equipa
        \item fundacaoEquipa - utilizamos um \emph{LocalDate} para se referir a data de fundaçao da equipa
        \item jogadores - que se refere a lista de jogadores que cada equipa tem, utilizamos uma \emph{List} de Jogador
    \end{itemize}

    Elaboramos os seus respetivos construtores bem como os seus \emph{getters} e \emph{setters} devido ao encapsulamento.
Nesta classe, desenvolvemos um método que nos permite obter o jogador através do seu número da camisola
\emph{obterJogadorPeloNumero} e um método para calcular o \emph{overall} da equipa. Para conseguirmos fazer a
transferência de jogador de uma equipa para outra, tivemos de implementar dois métodos tais como \emph{insereJogador} e
\emph{removeJogador}. Elaboramos também métodos que permitem coverter para \emph{string} a informação no
    ficheiro tais como o conteúdo da equipa, habilidades e posição. \par
    Por fim, estão definidos as funções \emph{toString}, \emph{equals} e \emph{clone}. \par


\section{Jogo}
    Na Classe Jogo desenvolvemos as seguintes variáveis:
    \begin{itemize}
        \item tempo - que se refere à data do jogo
        \item estado - que se refere ao estado do jogo
        \item equipa1 - que se refere à equipa visitada
        \item equipa2 - que se refere à equipa visitante
        \item golosVisitada - que se refere aos golos da equipa visitada
        \item golosVisitante - que se refere aos golos da equipa visitante
        \item ld - que se refere a data que o jogo ocorreu
        \item jogadoresEquipa1 - que se refere aos titulares da equipa1
        \item jogadoresEquipa2 - que se refere aos titulares da equipa2
        \item taticaEquipa1 - que se refere a tática da equipa visitada
        \item taticaEquipa2 - que se refere a tática da equipa visitante
        \item substituicoesEquipa1 - que se refere às substituições da equipa visitada
        \item substituicoesEquipa2 - que se refere às substituições da equipa visitante
    \end{itemize} \par
    Elaboramos os seus respetivos construtores bem como os seus \emph{getters} e \emph{setters} com o devido encapsulamento.
    Nesta classe é onde é calculado o resultado de um jogo, sendo este processo dividido em dois métodos
(\emph{simulacao\_part1} e \emph{simulacao\_part2}).\par
    O resultado de um jogo é obtido atraves do overall de uma equipa e por um determinado número de iterações que tem
como objetivo obter valores mais simpáticos. Além disso, estes métodos também tratam de fazer as substituições dos
jogadores em campo, algo que é realizado no intrevalo. \par
    Para além disso, também é possivel obter outras informações como a formação de determinada equipa bem como o resultado
final ou as substituções que foram efetuadas.


\section{Informacoes}
    Na Classe Informacoes desenvolvemos as seguintes variáveis:
    \begin{itemize}
        \item equipa - que se refere a todas as equipas
        \item jogadores - que se refere a todos os jogadores
        \item jogos - que se refere a todos os jogos
    \end{itemize} \par
    O objeto Informacoes tem como objetivo agrupar as diferentes informações num único objeto, assim conseguimos ter
todos os jogadores, equipas e jogos num único objeto. \par
    Para além disso, contém alguns métodos que são várias vezes usados ao longo do código para obter informações
relativas a um jogador/equipa/jogo.

\section{Parser}
    Elaboração desta classe permitiu a conversão da informação a partir da leitura do ficheiro

\section{Controller}
    Esta classe foi construida com o objetivo de estabalecer ligação entre a classe \emph{View} com a as restantes
classes.

\newpage

\section{View}
    Esta classe é encarregue de apresentar as informações de forma visual para o utilizador.
    Aparecendo assim:
    \begin{center}
        \vspace*{1cm}
        \includegraphics[width=8cm]{Screenshot from 2021-06-11 19-31-18}
    \end{center}

\section{FM}
Este objeto tem como objetivo compilar e correr o programa.\par

\subsection{Diagrama}
    \begin{center}
        \vspace*{1cm}
        \includegraphics[width=8cm]{Equipa}
    \end{center}

\newpage

\section{Conclusão}
    A realização deste projeto teve como objetivo por em prática a matéria lecionada nas aulas teóricas, bem como nas
aulas práticas. Tivemos diiculdade na implementação de certos métodos para a realização da simulação do jogo, devido às
várias abordagens que poderiamos ter.\par
    O nosso jogo tem a capacidade de realizar as funcionalidades mais básicas mas também implementamos métodos mais
complexos, bem como a simulação do resultado de um jogo podendo fazer alterações ao intervalo a ambas das equipas
e essas alterações têm impacto no resultado final.\par
    É possível também, criar um jogador e uma equipa de raiz, guardar e carregar estados e por fim aceder à informação
que vai sendo guardada ao longo das ações do utilizador.\par
    Em suma, estamos contentes com o nosso programa e com as suas funcionalidades, sentimos que estamos mais intrusados
em \emph{java} e que o progresso ao longo do semestre relativo à linguagem foi bastante positivo.


(...)

\end{document}
